\documentclass{article}
\usepackage{graphicx}
\usepackage{fancyhdr}
\usepackage{hyperref}
\usepackage{lipsum} %for some dummy text
\usepackage{array} %per gli array in tabella
\usepackage[table]{xcolor} %colori della tabella
\newcommand{\mytextbf}[1]{\textbf{\bfseries #1}}
\usepackage{tabularx}
\usepackage{geometry}
%variabili globali
\newcommand{\mainname}{GuessUJob}

\geometry{
    a4paper,
    total={170mm,257mm},
    left=20mm,
    top=20mm,
}

%di seguito imposterò i piè pagina
\pagestyle{fancy}
\fancyhf{}
\fancyfoot[L]{\thepage}
\fancyfoot[R]{\includegraphics[height=40pt]{job_seeking.png}}

%font utilizzato san serif
\renewcommand{\headrulewidth}{0pt} % rimuovi la linea dell'header
\renewcommand{\familydefault}{\sfdefault}

\begin{document}
\thispagestyle{empty}

\begin{center}%inizio documento
    \includegraphics[scale=0.6]{job_seeking.png}

    \vspace{1cm}

    \textbf{\huge{\mainname}} % Aggiunge il titolo

    \vspace{0.5cm}

    \textbf{\large Artificial Intelligence DOCUMENT}

    \textit{\large "Dipartimento di Informatica anno 2022/2023"}

    \textit{\large "Professore: Fabio Palomba"}

    \begin{table}[ht]
        \centering
        \begin{tabularx}{5.5cm}{l | r}
            \textbf{Autori}    & \textbf{Matricola} \\
            \hline
            Giulio Incoronato  & 0512111363         \\
            Antonio Mazzarella & 0512112830         \\
        \end{tabularx}
    \end{table}
\end{center}

\newpage %nuova pagina

\tableofcontents

\newpage

\section{Introduzione}
Quante volte hai avuto l'ansia di essere preso o pure no in uno specifico lavoro?
Quante volte ti sei domandato se fossi giusto tu per quel lavoro? Con la fine del proprio percorso
di studio ci si pongono tante domande e dubbi se si viene presi in un determinato lavoro oppure no.
\vspace{0.2cm}
\par
Tutto questo sorge perchè dopo diversi anni di studio si vuole avere la sicurezza di essere presi
in un determinato lavoro, magari il lavoro dei propri sogni. Sarebbe utile avere un tool in grado di
prevedere, attraverso dei dati, se sarai preso.
\vspace{0.2cm}
\par
Il nostro team mira a combattere tutte queste ansie creando un tool chiamato \textbf{"\mainname"}
che integrerà algoritmi di machine learning supervisionato che andranno ad analizzare un dataset
in modo da prevedere i risultati in base agli input forniti dall'utente.

\subsection{Link Utili}
\begin{enumerate}
    \item Questo è il link alla repository ufficiale di \textbf{\mainname}: \href{https://github.com/ShackWove/GuessUJob}{Link}
    \item Questo è il link dove abbiamo preso i dataset usati per l'addestramento: \href{https://www.kaggle.com/datasets/ahsan81/job-placement-dataset}{Link}
    \item Qui è dove è stata presa l'immagine: \href{https://www.flaticon.com/free-icon/job-seeking_1503438}{Link}
\end{enumerate}

\section{Business understanding}
\subsection{Obiettivi di business}
L'obiettivo principale di \textbf{\mainname} è la realizzazione di un tool con cui l'utente interagisce inserendo
dei dati chiesti in partenza sul suo percorso di studi, il tutto verrà analizzato e processato per poi dare in
output la previsione inerente al percorso di studio in questione. Lo scopo di questo tool sarà anche quello di
rimanere in un giudizio che non sarà in base al sesso della persona o altri tratti somatici.

\subsection{PEAS}
\begin{table}[ht]
    \centering
    \rowcolors{1}{gray!20!blue!5}{gray!20!blue!15}
    \begin{tabular}{| l | m{8cm} |}
        \hline
        \textbf{Performance} & Capacità dell'agente di prevedere se l'utente sarà preso o meno per un lavoro.                       \\
        \textbf{Enviroment}  & L'ambiente in cui l'agente opera rappresentato da un form di cui l'utente scriverà i dati necessari. \\
        \textbf{Actuators}   & Interfaccia utente dell'applicazione dove uscirà il valore predetto.                                 \\
        \textbf{Sensors}     & Form nell'interfaccia utente.                                                                        \\
        \hline
    \end{tabular}
\end{table}
\subsection{Proprietà dell'Ambiente}
L'ambiente possiede le seguenti proprietà:
\begin{itemize}
    \item \textbf{Completamente osservabile:} l'agente ha accesso completo a tutte le informazioni fornite dall'utente.
    \item \textbf{Deterministico:} lo stato dell'ambiente dipende dall'azione intrapresa dall'agente.
    \item \textbf{Sequenziale:} le decisioni dell'agente dipendono dagli input dell'utente.
    \item \textbf{Statico:} nel momento in cui l'agente sta elaborando la sua previsione l'utente non può modificare il form dato in partenza.
    \item \textbf{Discreto:} le previsioni dell'agente dipendono soprattutto dagli input inseriti dall'utente, oltretutto c'è un numero limitato e preciso di informazioni che l'utente può inserire.
    \item \textbf{Singolo-agente:} esiste solo un agente che opera nell'ambiente.
\end{itemize}

\end{document}
