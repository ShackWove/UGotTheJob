\documentclass{article}
\usepackage{graphicx}
\usepackage{fancyhdr}
\usepackage{hyperref}
\usepackage{lipsum}% for some dummy text
\newcommand{\mytextbf}[1]{\textbf{\bfseries #1}}

%di seguito imposterò i piè pagina
\pagestyle{fancy}
\fancyhf{}
\fancyfoot[L]{\thepage}
\fancyfoot[R]{\includegraphics[height=50pt]{job_seeking.png}}

\renewcommand{\headrulewidth}{0pt} % rimuovi la linea dell'header

\begin{document}
\thispagestyle{empty}

\begin{center}%inizio documento
    \includegraphics[scale=0.5]{job_seeking.png}

    \vspace{1cm}

    {\huge{GuessUJob}} % Aggiunge il titolo

    \vspace{0.5cm}

    {\large Artificial Intelligence DOCUMENT}
\end{center}

\newpage %nuova pagina

\tableofcontents

\newpage

\section{Introduzione}
\lipsum[3]
\lipsum[2]

\subsection{Seconda Subsection Introduzione}
\lipsum[1-3] % genera del testo fittizio da paragrafi 1 a 3

\subsubsection{Sottosezione della sottosezione}
\lipsum[1]

\begin{itemize} %{enumerate} per i numeri
    \item quella cosa rosa
    \item sopra nera nera e sotto
    \item di colore rosa
\end{itemize}

\begin{description}
    \item[Nome:] Mario Rossi
    \item[Età:] 30 anni
    \item[Professione:] Ingegnere
\end{description}

\begin{table}[ht]
    \centering
    \caption{Esempio di tabella flottante}
    \begin{tabular}{|l|c|r|}
        \hline
        Elemento sinistro & Elemento centrato & Elemento destro \\
        \hline
        1                 & 2                 & 3               \\
        4                 & 5                 & 6               \\
        7                 & 8                 & 9               \\
        \hline
    \end{tabular}
\end{table}


\end{document}